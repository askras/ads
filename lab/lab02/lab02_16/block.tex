\documentclass[tikz, margin=3mm]{standalone}
\usepackage[utf8]{inputenc}
\usepackage{tikz}
\usepackage[russian]{babel}

% \hyphenation
\usetikzlibrary{shapes.geometric, arrows, calc}
\tikzstyle{startstop} = [rectangle, rounded corners, minimum width=3cm, minimum height=1cm,text centered, draw=black, fill=red!30]
\tikzstyle{io} = [trapezium, trapezium left angle=70, trapezium right angle=110, minimum width=3cm, minimum height=1cm, text centered, draw=black, fill=blue!30]
\tikzstyle{process} = [rectangle, minimum width=3cm, minimum height=1cm, text centered, draw=black, fill=orange!30]
\tikzstyle{decision} = [diamond, minimum width=3cm, minimum height=1cm, text centered, draw=black, fill=green!30]

\tikzstyle{arrow} = [thick,->,>=stealth]


\begin{document}
    \begin{tikzpicture}[node distance=2cm]
        \node(start)[startstop]{НАЧАЛО};
        \node(loop_start)[process, below of = start, text width = 7cm]{Вычисляем хэш-сумму искомой строки};
        \node(i)[process, below of = loop_start]{i = 0};
        \node(max_find)[process, below of = i, text width=7cm]{Находим хэш-сумму подстроки текста длиной искомой строки взятой от i-го элемента};
        \node(condition_i)[decision, below of = max_find, yshift = -2cm, text width=4cm]{хэш-суммы искомой и текущей строк совпадают?};
        \node(condition_max)[decision, below of = condition_i, yshift = -4cm, text width=4cm]{Строки равны посимвольно?};
        \node(revert_1)[process, below of = condition_max, yshift = -2cm, text width = 7cm]{Записываем i, как один из ответов};
        \node(condition_0)[decision, below of = revert_1, yshift = -2cm, text width=4cm]{i = длина текста минус длина искомой строки?};
        \node(minus_i)[process, left of = condition_max, xshift = -3cm, yshift = 3cm]{i++};
        \node(stop)[startstop, below of = condition_0, yshift = -2cm]{КОНЕЦ};

        \coordinate(max_ind) at ($(max_find.north) + (0, 5mm)$);
        % \coordinate(i_cond) at ($(max_find.north) + (0, 5mm)$);

        \draw[arrow](start)--(loop_start);
        \draw[arrow](loop_start)--(i);
        \draw[arrow](i)--(max_find);
        \draw[arrow](max_find)--(condition_i);
        \draw[arrow](condition_i)-- node[anchor = west,  yshift = 1mm]{да}(condition_max);
        \draw[arrow](condition_i.east)-- node[anchor = south]{нет}+(2.6, 0) |-(condition_0);
        \draw[arrow](condition_max)-- node[anchor = west, yshift = 1mm]{да}(revert_1);
        \draw[arrow](condition_max.east)-- node[anchor = south]{нет}+(2.6, 0) |- (condition_0);
        \draw[arrow](revert_1)--(condition_0);
        \draw[arrow](condition_0.west)node[anchor = south, xshift = -10mm]{нет} -|(minus_i);
        \draw[arrow](minus_i)|-(max_ind);
        \draw[arrow](condition_0)-- node[anchor = west]{да}(stop);

    \end{tikzpicture}
\end{document}